\documentclass[11pt]{amsart}
\usepackage{graphicx}
\usepackage{amsmath}
\usepackage{fullpage}
\newtheorem{prop}{Proposition}
\newtheorem{corr}{Corollary}
\newtheorem{open}{Open Question}
\newtheorem{conj}{Conjecture}
\newcommand{\diag}{\textrm{diag}}
\title{Whirling Modes of the N-Chain}
\author{Warren Weckesser}
\address{Department of Mathematics, Colgate University}
\date{3 July 2003}
\begin{document}
\maketitle
\section{Introduction}
The \emph{n-chain} is a chain of spherical pendulums; topologically, the configuration
space is $(S^2)^{n}$.
We ignore friction, and
the only external force is gravity.

Background, etc.
Marsden and Scheurle \cite{MS} studied the double spherical pendulum, which in our terminology
is the $2$-chain.

\section{Equations of Motion and Relative Equilibria}

Let $\phi_i$ be the angle of the $i$th pendulum from the straight down position.

yada yada yada...



Let $\Phi = [\phi_1, \phi_2, \ldots, \phi_n]$,
$\tan\Phi = [\tan\phi_1, \tan\phi_2, \ldots, \tan\phi_n]$,
and $\sin\Phi$ is defined similarly.

\subsection{Equilibria}
\begin{prop}
The $n$-chain has $2^n$ equilibrium solutions
(with $\phi_i=0$ or $\phi_i = \pi$).
Only the equilibrium where $\phi_1 = \phi_2 = \cdots = \phi_n = 0$ is stable.
If an equilibrium has $k$ pendulums inverted, the linearization at that
equilibria has $k$ pairs of real eigenvalues, where the eigenvalues in each
pair have equal magnitudes and opposite signs.
\end{prop}
Let us denote an equilibrium as $\Phi_K$, such that
$\Phi_0 = (0,0,\ldots,0)$,
$\Phi_1 = (\pi,0,0,\ldots,0)$,
$\Phi_2 = (0,\pi,0,0,\ldots,0)$,
$\Phi_3 = (\pi,\pi,0,0,\ldots)$,
\ldots, and
$\Phi_{2^n-1} = (\pi,\pi,\pi,\ldots,\pi)$.
Thus, the binary representation of the numerical subscript
gives the configuration.  $\Phi_0$ is the stable solution in which all the
pendulums hang straight down.

\subsection{Relative equilibria}

\begin{prop}
Consider an equilibrium that has $k$ pendulums inverted.
Then $n-k$ families of relative equilibria bifurcate from this
equilibrium.
\end{prop}

\begin{corr}
The $n$-chain has $n2^{n-1}$ families of relative equilibria that
bifurcate from straight solutions.
\label{COR:COUNT}
\end{corr}

\begin{open}
Are there examples of an $n$-chain with relative equilibria that are not
in a family that bifurcates from a straight solution?
Or does Corollary \ref{COR:COUNT} given them all?
\end{open}

We now focus on the case where all the masses are equal and all the length are equal. 
Let $D_n = \textrm{diag}(n,n-1,\ldots,1)$, and let
\[
   A_n = \begin{pmatrix}
            n &   n-1 & n-2  &\ldots & 1 \\
            n-1 & n-1 & n-2  & \ldots & 1 \\
            n-2 & n-2 & n-2  & \ldots & 1 \\
            \vdots & & & \ddots &  \\
            1 & 1 & 1 & \ldots & 1
          \end{pmatrix}.
\]
%For example,
%\[
%   A_4 = \begin{pmatrix}
%              4 & 3 & 2 & 1 \\
%              3 & 3 & 2 & 1 \\
%              2 & 2 & 2 & 1 \\
%              1 & 1 & 1 & 1
%         \end{pmatrix}.
%\]
The equation for the relative equilibria may be written as
\begin{equation}
    \omega^2 A_n \sin(\Phi) = D_n\tan(\Phi).
    \label{EQN:RELEQ}
\end{equation}


\noindent
\emph{Bifurcations from the Trivial Solutions.}
The linearization of \eqref{EQN:RELEQ} at the stable equilibrium is
\[
   \omega^2 A_n \Phi = D_n \Phi,
\]
or
\[
   \left(\omega^2A_n - D_n\right)\Phi = 0.
\]
Nontrivial solutions bifurcate from the trivial solution when
\[
  \det\left(\omega^2A_n - D_n\right) = 0.
\]

A similar argument can be made for bifurcations from the unstable trivial solutions,
but we must take into account that while $\tan(\pi+h)\approx h$,
$\sin(\pi+h) \approx -h$.
Thus $\tan(\Phi_K + \delta\Phi) \approx \delta\Phi$,
and $\sin(\Phi_K+\delta\Phi) \approx S_K \delta\Phi$, where
$S_K = \diag( 1-2\Phi_{K,1}, 1-2\Phi_{K,2}, \ldots, 1-2\Phi_{K,n})$.
In other words, $S_K$ is an $n\times n$ diagonal matrix; the $j$th diagonal
elements is $1$ if $\phi_j = 0$ or $-1$ if $\phi_j = \pi$.
With this notation, the formula for the values of omega at which nontrivial
solutions bifurcate from $\Phi_K$ is
\begin{equation}
   \det\left(\omega^2A_n S_K - D_n\right) = 0.
\end{equation}

\section{Asymptotic Whirling Modes}
We consider the solutions to \eqref{EQN:RELEQ} in the limit $\omega\rightarrow\infty$.
We rewrite \eqref{EQN:RELEQ} as
\begin{equation}
    A_n \sin(\Phi) = D_n\tan(\Phi)/\omega^2.
    \label{EQN:RELEQ2}
\end{equation}
Consider what may happen to an angle $\phi_i$ as $\omega$ becomes very large.
If $\phi_i$ approaches an angle other than $\pi/2$ or $-\pi/2$, then
$\tan\phi_i$ approaches a finite number, so $(\tan\phi_i)/\omega^2$ approaches
zero.  On the other hand, if $\phi_i$ approaches $\pm\pi/2$,
then $\sin\phi_i$ approaches zero, and
$\tan\phi_i$ becomes infinite; this means $(\tan\phi_i)/\omega^2$
\emph{may approach a nonzero number} as $\omega$ increases.

The following conjecture is based on numerical studies.
It applies to the case of all masses and lengths equal.
\begin{conj}
Consider a family of relative equilibria that bifurcates from
a straight equilibrium.  Then for each angle $\phi_i$,
$\lim_{\omega\rightarrow\infty} \phi_i$ is either $0$, $\pm \pi/6$ or $\pm \pi/2$.
\end{conj}

\section{Linear Stability of the Asymptotic Whirling Modes}

\section{Conclusion}

\begin{thebibliography}{99}
%\bibitem{ANTMAN} S.~Antman,
%   \emph{Nonlinear Problems in Elasticity},
%   Springer-Verlag, New York (1995).
%\bibitem{KOLODNER} I.~I.~Kolodner,
%   Heavy rotating string--a nonlinear eigenvalue problem,
%   \emph{Comm.~Pure~Appl.~Math.} \textbf{8}, 395--408 (1955).
\bibitem{LEWIS} D.~Lewis,
   Stacked Lagrange tops,
   \emph{J. Nonlinear Science} \textbf{8} 63--102 (1998).
\bibitem{MARSDEN} J.~E.~Marsden, \emph{Lectures on Mechanics},
   Cambridge Univ.~Press, Cambridge (1992).
\bibitem{MR} J.~E.~Marsden, T.~S.~Ratiu,
   \emph{Introduction to Mechanics and Symmetry},
   Springer Verlag, New York (1999).
\bibitem{MS} J.~E.~Marsden and J.~Scheurle,
   Lagrangian reduction and the double spherical pendulum,
   \emph{Z. angew Math.~Phys.} \textbf{44}, 17--43 (1993).
%\bibitem{REEKEN1} M.~Reeken,
%    The equation of motion of a chain,
%    \emph{Math.~Z.}, \textbf{155}(3), 219--237 (1977).
%\bibitem{REEKEN2} M.~Reeken,
%    Classical solutions of the chain equation. (I,II),
%    \emph{Math.~Z.}, \textbf{165}(2), 143--169;
%                     \textbf{166}(1), 67--82
%    (1979).
%\bibitem{REEKEN3} M.~Reeken,
%    The rotating string,
%    \emph{Math.~Ann.} \textbf{268}(1), 59--84 (1984).
%\bibitem{TOLAND} J.~F.~Toland,
%    On the stability of rotating heavy chains,
%    J.~Diff.~Eqns., \textbf{32}, 15--31 (1979).
%\bibitem{STUART} C.~F.~Stuart,
%    Spectral theory of rotating chains,
%    \emph{Proc.~Roy.~Soc.~Edin.}, \textbf{73A},
%    199--214 (1975).
%\bibitem{WATSON} G.~N.~Watson,
%   \emph{A Treatise on the Theory of Bessel Functions},
%   Cambridge Univ. Press, Cambridge (1922).
\end{thebibliography}
\end{document}
